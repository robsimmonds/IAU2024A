\NeedsTeXFormat{LaTeX2e}

\documentclass{iau_FM}

\pagenumbering{gobble}

\usepackage{graphicx}
\usepackage{amsmath,amssymb,latexsym,bm,mathrsfs,sansmath,array}
\usepackage{hyperref}
%\usepackage{natbib}
\usepackage[natbib=true]{biblatex}
\addbibresource{rob.bib}

\title{Ilifu: A Cloud Computing System Supporting Astronomy and Bioinformatics}
\DeclareUnicodeCharacter{202F}{\,}

\author{Rob Simmonds}

\affiliation{Inter-university Institute for Data Intensive Astronomy \\ South Africa}
\pubyear{2024}
\setcounter{page}{1}
\jname{Astronomy in Focus, Focus Meeting 7} 
\editors{S. S. Mohamed, A. R. Taylor and B. Vaidya}

%\date{}

%\graphicspath{ {./IMAGES/} }

\begin{document}

\maketitle

\begin{abstract}
This paper describes the {\sc ilifu} cluster which is an OpenStack based cloud system located at the University of Cape Town. 
It is operated by the Inter-university Institute for Data Intensive Astronomy and mainly supports 
astronomy and bioinformatics research. In terms of Astronomy the main data processed is from the MeerKAT and MeerLicht 
telescopes that are both based in South Africa.
 \end{abstract}
 \noindent {\bf Keywords:} OpenStack, IaaS cloud, Astronomy, Bioinformatics

\section{Introduction}
 
The {\sc ilifu} cluster was established in 2016 and 
is operated by the Inter-university Institute for Data Intensive Astronomy (IDIA) which is an
institute that spans three universities; the University of Cape Town (UCT), 
the University of the 
Western Cape (UWC) and University of Pretoria (UP).
The cluster was built using the management nodes from an 
existing OpenStack (https://openstack.org/) system that was constructed to explore the 
use of an OpenStack based Infrastructure of a Service (IaaS) cloud system to support university research. 
This meant the
system could be brought into production very quickly as it just needed the new
network switches, the compute and storage nodes added to the working OpenStack system.
The name ilifu means
Cloud in the local Xhosa language and the system is now used primarily to support
astronomy and bioinformatics research.

The rest of this paper is laid out as follows. Section~\ref{sec:hardware}
 explains the hardware infrastructure and how
the system is connected to the research networks. Section~\ref{sec:middleware} 
explains the middleware employed while
Section~\ref{sec:core_software} describes the core software that is run.
Section~\ref{sec:gateway} then explains the IDIA
science gateway that was added to the system in 2023. Section~\ref{sec:users}
describes the main uses of the system and 
the paper is concluded and future work is discussed in
Section~\ref{sec:conclusions}.



% Hardware architecture
% Middleware
% Science Gateway
% Userbase
% Astro projects
% Bioinformatics projects
% The path forward


\section{Hardware}
\label{sec:hardware}



The ilifu cluster currently has over 4000 processor cores with most nodes having 256\,GB of RAM and 32 cores. 
Some nodes have 512\,GB 
and one has 1.5\,TB of RAM. The large memory node is mainly used for remote visualisation 
 and testing of Virtual
Reality (VR) streaming. The cluster has over 13 Petabytes of raw storage which is mainly provided
 by hard drives, but 
does include SSDs to provide backing store for virtual machines (VMs) and provide
POSIX shares used mainly
to support visualisation. The main
 cluster interconnect is a mix of 50\,Gb/s and 100Gb/s Ethernet using Mellanox SN2700 switches, 
 with the storage servers connected at 100\,Gb/s and the 
most of compute servers connected at 50\,Gb/s. The cluster is connected to 
the SANReN (https://saran.ac.za) research network with a 100\,Gb/s link
that then connects to a 400\,Gb/s link within South Africa and out to Europe. 
The network uplink connects 
to SANReN at the main point of presence in Cape Town and is not shared
with any of the UCT campus traffic.


\section{Middleware}
\label{sec:middleware}

{\sc Ilifu} utilises OpenStack IaaS middleware that allows us to run VMs
 or bare-metal deployments
utilising the OpenStack Ironic subsystem. Recently we have also been experimenting with 
Kubernetes (K8s) (https://kubernetes.io/)
as that is an interesting way of supporting interactive applications and for deployment
in general, in particular for micro-service architecture applications. We have not promoted K8s for
 production use yet, since we are still learning the best ways to have it interact with our
 other low level environments.
We use Ceph to provide storage. This provides ephemeral storage to back VM
block storage to create volumes (mainly on SSDs), Object Storage for supporting projects
with large numbers of small files and
CephFS shares for creating large cluster file-systems. All the storage is configurable from the 
OpenStack Horizon dashboard enabling users to create their own storage entities
and this also makes it very fast for the support team to create new computing
and storage environments when needed.

Some projects have long-lived cluster environments within ilifu which they maintain themselves using
the OpenStack tools.
Examples of this is the cluster that is used to process the data from the MeerLicht
 (https://science.uct.ac.za/meerlicht) telescope which
is an optical telescope that tracks the MeerKAT~\cite{meerkat} telescope's observations. 
This functionality
is also employed
by some of the bioinformatics projects to support custom environments and used to add
custom environments to support specific workshops and hackthons. The latter was done for the
 FM7 hackathons held during the 2024 IAU General Assembly in Cape Town (see FM7 summary, this volume). 

Authentication and authorisation are handled by KeyCloak (https://keycloak.org). 
This system enables us 
to use local passwords,
or preferably to use federated identity providers that can be accessed through SAFIRE 
(https://safire.ac.za/), 
EduGain (https://edugain.org/) and EGI's Check-in (https://egi.eu/service/check-in/).
 It also provides Single Sign-on (SSO)
to all our 
web based applications when they are accessed through the 
IDIA Science Gateway (see Section~\ref{sec:gateway}).
Enabling users to authenticate using external identity providers means that support staff no longer need
to reset passwords for users. This is something that took time for them to do and also could take
time for them to get to if the service was required at a weekend. Most institutional password
management systems provide a way for users to reset their passwords without needing assistance,
so we leave that to those systems to deal with.

\section{Core Software}
\label{sec:core_software}

A large amount of software is provided on ilifu, some of which is specific to domains such as {\sc CASA}
(https://casa.nrao.edu) based
astonomy pipeline~\cite{pipelinepaper} 
developed by IDIA that is used to process most of the MeerKAT data that arrives
at {\sc ilfu}. As far as possible application software is managed in Singularity containers enabling users
to access a large number of applications and often multiple versions of the software. Other software, such
as JupyterHub (https://jupyter.org/hub) support many domains, in this case by 
supporting multiple
 kernels that then can run 
specific notebooks for specific tasks.

Interactive web based applications that are accessible directly from external networked devises include
the CARTA astronomy visualisation system (http://cartavis.org) and Gen3 (http://gen3.org) bioinformatics system. 
Note that JupyterLab that starts
JupyterHub instances is also available in this way.


\section{IDIA Science Gateway}
\label{sec:gateway}

As part of a project funded by The Data Intensive Research Initiative of South Africa (https://dirisa.ac.za), 
IDIA created a science gateway to access {\sc ilifu}. The aim was to create
a portal that would provide a single sign-on environment to best-in-class web applications with unified branding. 
This aim was based on 
previous experience creating the CyberSKA~\cite{cyberska} portal that was run in
 Canada and South Africa for over fifteen 
years which
was created using a portal toolkit. The problem with that was that many of the components needed to written and maintained
by the team that created the portal. Since CyberSKA used many software 
layers, maintaining the components was very time consuming, and was necessary when security issues were discovered
somewhere in the stack.
Seeking funding for code maintenance is more difficult than gaining funding 
for new tools, so we wanted to limit the number of
components that needed to be maintained by ourselves.

Figure~\ref{fig:gateway_design} shows the basic design diagram for the gateway. This was created
quickly before an IDIA group meeting on a Monday morning. The inital version of the gateway was working one week later and
exposed to the end users in production form three weeks after this. That compares to years of work that went
 into creating the CyberSKA gateway
and shows the power of the tools used to create the gateway.

\begin{figure}
    \centering
    \includegraphics[scale=0.3]{gateway-design01b.png}
    \caption{High level design of the IDIA Science Gateway.}
    \label{fig:gateway_design}
\end{figure}

A few of the more simple components were created by our support team using the Django 
(https://www.djangoproject.com)
portal framework.
These included the Landing Page, an SSH public 
key management page and a Usage Information
Page. The purpose of the SSH public key management page was to enable users to add new public keys if
there was any problem with their current key pair. This was a task previously requested from the
support team and happened quite often.

Other applications are accessed via an OpenOnDemand (https://openondemand.org) instance that provided fast
setup and easy configuration (see Figure~\ref{fig:gateway-launch-page}). 
This was the major key to being able to setup the IDIA Science 
Gateway so quickly.
Of the best in class applications JupyterLab is exposed from the OpenOnDemand Dashboard. Also exposed in this
way are the CARTA (https://cartaviz.org) visualisation system, the Globus (https://globus.org)
Wide Area Network transfer client interface and the OpenStack Horizan
Dashboard that is used for configuring the computing environments. Also this year we added access
to the Singularity Registry (https://singularityhub.github.io/sregistry)
 that is used to maintain the Singularity 
containers that provide the majority of software to the {\sc ilifu} cluster and allows containers to be downloaded
for use on other systems.

The Data Release pages are exposed using the Wagtail (https://wagtail.org) Content Management System (CMS)
that is often used with Django. This is exposed from the Front Page, for access to only public data,
and from within OpenOnDemand to provide access to all the released data.


%\includegraphics[scale=.18]{gateway01.png}


\begin{figure}
    \centering
    \includegraphics[scale=.17]{gateway04.png}
    \label{fig:gateway-launch-page}
\caption{The IDIA Science Gateway application launch page.}
\end{figure}
    

%Link in to SAFIRE / Edugain to remove the need to manage user passwords
%Limit the amount of user support required
%Spent about a year experimenting with different tools that could help achieve these goals
%Had many problems with tools being hard to install, not working as advertised or being unreliable
%Big improvements in certain tools during the first half of 2023 left us ready to work on a deployment

%Diagram.
%Story of how we got it working so quickly.

%\includegraphics[scale=.9]{gen3.png}

%Include some diagrams from the talk.


\section{Users}
\label{sec:users}

Five of the eight MeerKAT telescope Large Survey projects are run on ilifu. 
These include:
\begin{itemize}
\item LADUMA~\cite{laduma} (Deep neutral hydrogen)
\item MIGHTEE~\cite{mightee} (Deep continuum imaging of the early universe)
\item MeerKAT Fornax Survey~\cite{formax} (Deep HI Survey of the Fornax cluster)
\item MHONGOOSE~\cite{mongoose} (targeted nearby galaxies HI)
\item ThunderKAT~\cite{thunderkat} (exotic phenomena, variables and transients)
\end{itemize}

Research associated with 
77 MeerKAT Open Time projects and over 20 Director Discretionary Time projects have also been supported
on ilifu to date.

%Talk about multiple countries.
%Can take counts from the pie charts in talk.

The ilifu cluster was partly funded by the Cbio~\cite{cbio}
biochemistry group at UCT.
There are currently about 140 active bioinformatics users and 
current projects include H3ABioNet~\cite{H3ABioNet}
and eLwazi (https://elwazi.org/),
an African led open data science platform that provides an interactive environment 
to apply data science techniques to 
diverse datasets for novel health discoveries.
IDIA staff have worked on updating the Gen3 (http://gen3.org) bioinformatics platform 
to support eLwazi in the OpenStack
middleware environment.


\section{Summary and Future Work}
\label{sec:conclusions}

%Would like updated national / international authorization service

The ilifu OpenStack cluster that was established in 2015 supports over 550 regular users working mainly in Astronomy
and Bioinformatics. The IDIA Science Gateway has been added that enables signon from federated identity providers, and 
in conjunction with KeyCloak, provides SSO within the cluster. The use of the OpenStack middleware
has enabled a very flexible environment, where custom clusters with their own storage connected can be
created quickly. This provides the ability to support projects that would be difficult to support in
a conventional high performance computing environment without far more interaction for the cluster's
support team and without delays caused by needing lower level configuration.

Currently for users of MeerKAT data to get the data to ilifu they need to log into a portal at
South African Radio Astronomy Observatory (SARAO) (https://sarao.ac.za)
and register the need for the transfer. We would like to streamline that so
they can either do that directly from the IDIA Gateway, or SSO from the IDIA Gateway
to SARAO register the transfer. We are currenly looking at how we could best use data replication tools, 
in particular
RUCIO (https://rucio.cern.ch) to better schedule the movement of data between SARAO, 
IDIA and possibly other sites such 
as Centre for High Performance Computing (https://chpc.ac.za) in Cape Town. 
We see the need for that to avoid the case where 
we have the same raw data available
on multiple systems that could be accessed quickly and avoid users wanting to keep copies locally
due a lack of clarity on when data will be fetched. This will be aided by planned changes to the
way that SARAO stores data, but improving automation in this area will help in any case.

Other changes to ilifu itself could be driven by changes in the applications being run.
For example, IDIA is currently looking at changing the design of the backend of the 
CARTA visualisation system from a monolithic to a micro-services architecture that could
provide greater scalability on large clusters and supporting that well will require
some changes to the ilifu middleware.

We would also like to federate more functions with more sites to enable users to move quickly
from one cluster to another, or just offload tasks across clusters more simply. We will
explore funding options to enable this.

%Planning to deploy data replication system to ease tracking of data products
%Many extensions to CARTA planned
%Improved collaboration features
%Experimenting with micro-services architecture
%Improved deployment for HPC and K8s systems
%Many additional science features
%Need to extend the Bioinformatics Science Gateway
%Need to extend IDIA pipeline to handle MeerKAT+ heterogenous array

%\bibliographystyle{plain}
%\bibliography{rob.bib}

\printbibliography

\end{document}