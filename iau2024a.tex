
\documentclass[12pt,twocolumn,a4paper]{article}

\usepackage{graphicx}

\title{ilifu: A cloud computing system supporting Astronomy and Bioinformatics}


\author{Rob Simmonds}
\date{}

\begin{document}

\maketitle

\begin{abstract}
   
This paper describes the ilifu cluster which is an OpenStack based cloud system located at the University of Cape Town. 
It is operatated by the Inter-university Institute for Data Intensive Astronomy and mainly supports 
astonomy and bioinformatics research. In terms of Astronomy the main data processed is from the MeerKAT and MeerLicht 
telescopes that are both based in South Africa.

 \end{abstract}

 Keywords: OpenStack, IaaS cloud, Astronomy, Bionformatics

\section{Introduction}
 
The ilifu cluster was established in 2016 and 
is operated by the Inter-university Institute for Data Intensive Astronomy (IDIA) which is an
institute that spans three universities; the University of Cape Town, the University of the 
Western Cape and University of Pretoria.
The cluster was built using the management nodes from an 
existing OpenStack system that was constructed to explore the 
use of an Openbstack based Infrastructre of a Service (IaaS) cloud system to support university research. 
This meant the
system could be brought into production very quickly since as it basically just needed the new
network switches, the compute and storage nodes added to the working OpenStack system.
The name ilifu means
Cloud in the local Xhosa languague and the system is now used primarily to support
astronomy and bioinformatics research.

The rest of this paper is laid out as follows. Section~\ref{sec:hardware} explains the hardware infrastructure and how
the system is connected to the research networks. Section~\ref{sec:middleware} explains the middleware employed with 
Section\ref{sec:core_software} describes the core software that is run. Section~\ref{sec:gateway} then explans the IDIA
science gateway that was added to the system in 2023. Section~\ref{sec:users} Finally the paper is concluded and future work is discussed In
Section~\ref{sec:conclusions}.



% Hardware architecture
% Middleware
% Science Gateway
% Userbase
% Astro projects
% Bioinformatics projects
% The path forward


\section{Hardware}
\label{sec:hardware}



The ilifu cluster currently has over 4000 processor cores and over 13 Petebytes of raw storage. The main
 cluster interconnect is a mix of 50Gb/s and 100Gb/s Ethernet, with the storage servers connected at 100Gb/s and the 
most of compute servers connected at 50Gb/s. It is connected to the SANReN research network with a 100Gb/s link
that then connects to a 400Gb/s link that withing South Africa and out to Europe.


\section{Middleware}
\label{sec:middleware}

Utilizes OpenStack Infrastructure as a Service middleware
Utilizes Ceph to provide storage
Ephemeral storage to back VMs
Block storage to create volumes (mainly on SSDs) Object storage (S3) for applications with very large numbers of files
CephFS shares for creating large cluster file-systems Storage all configurable from the OpenStack Horizon dashboard

Provides easy way to create custom storage configurations for projects

\section{Core Software}
\label{sec:core_software}


\section{IDIA Gateway}
\label{sec:gateway}

As part of a project funded by DIRISA~\cite{DIRISA2015} IDIA created a science gateway to accass ilifu. The aim was to create
a portal that would provide a single signon environment to best-in-class web applications with unified branding. 
This aim was based on 
previous expirance creating the CyberSKA~\cite{cyberska} portal that was run in Canada and South africa that
was created using a web toolkit. The problem with that was that many of the components needed to written and maintained
by the team that created the portal. Since CyberSKA used many software 
layers maintaining the components was very time consuming, and was necessary when security issues were discovered
somewhere in the stack.
Seeking funding for code mainenance is more difficult that gaining funding for new tools, so we wanted to limit the number of
components that neeeded to be maintained by ourselves.

Figure~\ref{fig:gateway_design} shows the basic design diagram for the gateway. This was created q
quickly before an IDIA group meeting on a Monday morning. The inital version of the gateway was working one week later and
exposed to the end users three weeks after this. That comepares to years of work that went into creating the CyberSKA gateway
and shows the power of the tools used to create the gateway.


\begin{figure}
    \includegraphics[scale=0.3]{gateway-design01.png}
    \caption{}
    \label{fig:gateway_design}
\end{figure}


Main Portal
Django
FrontPage
Public Data (using Wagtail CMS)
SSH Public Key management
Usage information
Singularity Registory.
Initial container listing (management still done at CLI level for now)
Open OnDemand
Launch site for authenticated tools
Other tools
Keycloak – AAI : Provides access to SAFIRE / EduGain
Jupyter – Notebooks
CARTA – Remote Visualization
Globus – WAN file transfer
OpenStack Dashboard – IaaS compute and Storage control



%\includegraphics[scale=.18]{gateway01.png}

\begin{figure}
\includegraphics[scale=.09]{gateway02.png}
\end{figure}

\begin{figure}
    \includegraphics[scale=.14]{gateway01.png}
\end{figure}
    

Link in to SAFIRE / Edugain to remove the need to manage user passwords
Limit the amount of user support required
Spent about a year experimenting with different tools that could help achieve these goals
Had many problems with tools being hard to install, not working as advertised or being unreliable
Big improvements in certain tools during the first half of 2023 left us ready to work on a deployment

Diagram.
Story of how we got it working so quickly.

\includegraphics[scale=.9]{gen3.png}




Include some diagrams from the talk.


\section{Users}
\label{sec:users}

IDIA supports 5 of the 8 MeerKAT Large Survey projects, and has supported research associated with 77 MeerKAT Open Time projects and over 20 Director Discretionary Time projects to date

MeerKAT Large Survey Projects
LADUMA (Deep neutral hydrogen)
MIGHTEE (Deep continuum imaging of the early universe)
MeerKAT Fornax Survey (Deep HI Survey of the Fornax cluster)
MHONGOOSE (targeted nearby galaxies HI)
ThunderKAT (exotic phenomena, variables and transients)

Talk about multiple countries.
Can take counts from the pie charts in talk.

ilifu cluster partly funded by the Cbio biochemistry group at UCT
Currently about 140 active bioinformatics users
Current projects include eLwazi (https://elwazi.org/)
African led open data science platform that provides an interactive environment to apply data science techniques to diverse datasets for novel health discoveries.
IDIA staff have worked on updating the Gen3 bioinformatics platform to run in OpenStack


\section{Summary and Future Work}
\label{sec:conclusions}

Would like updated national / international authorization service
Planning to deploy data replication system to ease tracking of data products
Many extensions to CARTA planned
Improved collaboration features
Experimenting with micro-services architecture
Improved deployment for HPC and K8s systems
Many additional science features
Need to extend the Bioinformatics Science Gateway
Need to extend IDIA pipeline to handle MeerKAT+ heterogenous array



\end{document}